\documentclass[11pt]{article}
\usepackage{hyperref}
\usepackage{amsmath}
\usepackage{amssymb}
\DeclareMathOperator*{\argmin}{arg\,min}

\title{\textbf{Transfer}}

\author{Mark McLeod}
\date{}
\begin{document}

\maketitle

\section{Paper Section}
\subsection{Problem outline}
material from \url{https://github.com/markm541374/tariffset/blob/master/problem_spec.pdf}
\subsection{Thermal agent maths}
Analysis of the individual agent optimisation and QP solution.

Material from \url{https://github.com/markm541374/tariffset/blob/master/quadthermo.pdf}
\subsection{GPGO}
Justification of using an expensive optimisation technique rather than standard ones.
\subsection{Results}
hopefully showing that the load responds to the tariff and that GPGO makes good reductions in the objective fn of the load.

\section{Review/Proposal}

\subsection{Review-optimisation}
Surrogate surface techniques. GPs. different maximization options, EI, PI, Entropy. multistep lookahead.
\subsection{Review-tariffsetting}
Smartgrid overview. Other proposals for load management. This method is new because it both publishes ahead of time (giving agents time to plan ahead and b more flexible) and is not a direct exposure of market prices so is a proper control signal.

\subsection{Proposal-model}
Form a defensible model of a home using gov. surveys, power company stats, dissagregation datasets. Correlations between insulation/occupancy/location/weather/season/day-to-day.
\subsection{Proposal-optimisation}
Fast two-step lookahead \url{http://www.robots.ox.ac.uk./~markm/MSGPGO.pdf}

Using previous result as information for next day, using true load each day to update the model.

\end{document}