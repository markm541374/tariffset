\documentclass[11pt]{article}

\usepackage{amsmath}
\usepackage{amssymb}
\DeclareMathOperator*{\argmin}{arg\,min}

\title{\textbf{Expressing a thermostat as a quadratic programing problem}}

\author{}
\date{}
\begin{document}

\maketitle

\section{}

I want to find a method to find the power consumption of a smart thermostat given a tariff profile and a model of the home the thermostat is responsible for.

The home is modelled as a simple thermal system. It has thermal mas $cm$ insulation $\frac{1}{k}$ and a heater which when on outputs power $P$. If the external temperature is $T^{e}(t)$ then the temperature is governed by the first order differential equation
\begin{equation}
\frac{dT(t)}{dt} = -\frac{k}{cm}(T(t)-T^{e}(t)) +\frac{P\delta(t)}{cm}
\end{equation}
where $\delta(t) \in (0,1)$ determines the power setting of the heater. For a sufficiently small timestep $\Delta t$  this can be approximated as an update rule
\begin{equation}
T_{n+1} =  T_{n}-\frac{\Delta t k}{cm}(T_{n}-T_{n}^{e}) + \frac{\Delta t P \delta_{n}}{cm} 
\end{equation}
If the time period under consideration is discretized into $N$ periods of $\Delta t$ then $T, T^{e},\delta$ can be represented as column vectors of size $N+1$ and the update rule can be expressed as a matrix equation and simplified to give
\begin{equation}
\left( \mathbf{I}_{N+1 \times N \times 1}-(1-\frac{\Delta t k}{cm})\mathbf{U}\right) \underline{T} = \frac{\Delta t k}{cm}\mathbf{U}\underline{T}^{e}+\frac{P}{cm}\underline{\delta}+\underline{\Gamma}
\end{equation}
\begin{equation}
\underline{T}=\mathbf{\Phi} \underline{\delta}+\underline{\Psi}
\end{equation}
where
\begin{equation}
\mathbf{\Phi}= \frac{P}{cm} \left( \mathbf{I}_{N+1 \times N \times 1}-(1-\frac{\Delta t k}{cm})\mathbf{U}\right)^{-1}
\end{equation}
\begin{equation}
\underline{\Psi}= \left( \mathbf{I}_{N+1 \times N \times 1}-(1-\frac{\Delta t k}{cm})\mathbf{U}\right)^{-1} \left( \frac{\Delta t k}{cm}\mathbf{U}\underline{T}^{e} + \underline{\Gamma}\right)
\end{equation}
and
\begin{equation}
\mathbf{U} = \left[
\begin{array}{c|c}
\underline{0}_{1\times N} & 0 \\ \hline
\mathbf{I}_{N} & \underline{0}_{N \times 1}
\end{array}\right]
\end{equation}
\begin{equation}
\underline{\Gamma} = \left[
\begin{array}{c}
T_{0} \\ \hline
\underline{0}_{N \times 1}
\end{array}\right]
\end{equation}
Thus the given all the required environmental constants the temperature profile has been expressed as an affine function of the control input.
\section{}
There are to costs associated with a given control input. The actuation cost is $\underline{\Lambda}^{T} \underline{\delta}$ where the $\Lambda_{n}$, is the unit price of electricity at $t=n \Delta t$. The utility cost is $(\underline{T}-\underline{T}^s)^{T}\mathbf{Q}(\underline{T}-\underline{T}^s)$ where ${T}^{s}_{n}$ is the desired temperature at time $n \Delta t$ and $\mathbf{Q}=\mathrm{diag}(\underline{q})$ where $q_{n}$ is the coefficient of quadratic cost of deviation from the desired temperature at the same time.

The quantity to be minimised is therefore
\begin{equation}
f = (\underline{T}-\underline{T}^s)^{T}\mathbf{Q}(\underline{T}-\underline{T}^s)+\underline{\Lambda}^{T} \underline{\delta}
\end{equation}
Rather than having a very high dimensional $\underline{\delta}$ constrained to be either zero or one which is difficult to solve and very high dimensional we reduce the dimension of the input to $M$ by defining
\begin{equation}
\underline{\delta}=\mathbf{D}\underline{u}
\end{equation}
where
\begin{equation}
\mathbf{D}=\mathbf{I}_{M} \otimes \underline{1}_{J \times 1},\qquad N=MJ
\end{equation}
which gives
\begin{equation}
f = (\mathbf{\Phi} \mathbf{D}\underline{u}+\underline{\Psi}-\underline{T}^s)^{T}\mathbf{Q}(\mathbf{\Phi} \mathbf{D}\underline{u}+\underline{\Psi}-\underline{T}^s)+\underline{\Lambda}^{T} \mathbf{D}\underline{u}
\end{equation}
where instead of the on/off state over a short time interval $u$ now defines the mark/space ratio of a PWM system over a much longer period and so power consumption and temperatures are now true in expectation rather than in value. With some substitutions this gives a new function to be minimised
\begin{equation}
f'=\underline{u}^{T}\mathbf{R}\underline{u}+\underline{S}^{T}\underline{u}
\end{equation}

where
\begin{equation}
\mathbf{R}=\mathbf{D}^{T}\mathbf{\Phi}^{T}\mathbf{Q}\mathbf{\Phi}\mathbf{D}
\end{equation}
\begin{equation}
\underline{S}^{T}=2(\underline{\Psi}-\underline{T}^s)^{T}\mathbf{Q}\mathbf{\Phi}\mathbf{D}+\underline{\Lambda}^{T} \mathbf{D}
\end{equation}
\section{}
There are many possible constraint sets to give desired behavior. The minimum required constraint is that the control signal must be between zero and one at all times
\begin{equation}
0 \leq \delta_{n} \leq 1 \qquad \forall n
\end{equation}
This is expressed in standard form for the quadratic problem as
\begin{equation}
\left[
\begin{array}{c}
\mathbf{I}_{M} \\ \hline
- \mathbf{I}_{M} 
\end{array}\right]\underline{u} \leq \left[\begin{array}{c}
\underline{1}_{M \times 1} \\ \hline
\underline{0}_{M \times 1}
\end{array}\right]
\end{equation}
A constraint that is likely to be used is a maximum or minimum temperature requirement. $T_{n} \geq T^{min}$ or $T_{n} \leq T^{max}$. This is expressed in terms of $\underline{u}$ as
\begin{equation}
\underline{e}_{N+1,n}^{T}  \mathbf{\Phi D} \underline{u} \leq T^{max}-\underline{e}_{n}^{T} \underline{\Psi}
\end{equation}
\begin{equation}
-\underline{e}_{N+1,n}^{T} \mathbf{\Phi D} \underline{u} \leq -T^{min}+\underline{e}_{n}^{T} \underline{\Psi}
\end{equation}
Where $\underline{e}_{M,i}$ is the column vector with $M$ elements with 1 at position $i$ and zero elsewhere.
and the budget constraint
\begin{equation}
\Lambda^{T} \mathbf{D} \underline{u} \leq \beta
\end{equation}
This leads to a full set of inequality constraints
\begin{equation}
\left[
\begin{array}{c}
\mathbf{I}_{M} \\ \hline
- \mathbf{I}_{M} \\ \hline
\underline{e}_{N+1,n}^{T}  \mathbf{\Phi D}  \\
\vdots \\ \hline
-\underline{e}_{N+1,n}^{T}  \mathbf{\Phi D}  \\ 
\vdots \\ \hline
\Lambda^{T} \mathbf{D}
\end{array}\right]
\underline{u}\leq \left[
\begin{array}{c}
\underline{1}_{M \times 1} \\ \hline
\underline{0}_{M \times 1} \\ \hline
T^{max}_{n}-\underline{e}_{n}^{T} \underline{\Psi} \\
\vdots \\ \hline
-T^{min}_{n}+\underline{e}_{n}^{T} \underline{\Psi} \\ 
\vdots \\ \hline
\beta
\end{array}\right]
\end{equation}
Another potentially useful constraint is to specify a mean temperature over a time period. If $\underline{\alpha}$ is a column vector of size $N+1$ with ones at the positions corresponding to times that are to be averaged over and zeros elsewhere, and $T^{avg}$ is the desired mean over that period then the standard form constraint is
\begin{equation}
\underline{\alpha}^{T} \mathbf{\Phi} \mathbf{D} \underline{u} = \|\underline{\alpha}\|_{0} T^{avg} -\underline{\alpha}^{T} \underline{\Psi}
\end{equation}

\end{document}